Dans ce projet, nous nous proposons de traiter la tâche de reconnaissances d’entités nommées qui est une sous-tâche de l’extraction d’informations. La reconnaissance d’entités nommées consiste à détecter une entité textuelle (un mot, ou un groupe de mots) et à la classer dans une catégorie pouvant être des noms de personnes, de lieux, d’organisations, ou d’autres catégories plus spécifiques. Dans notre cas, nous nous intéressons à la reconnaissance de noms communs issus de la littérature scientifique médicale. 

La quantité ainsi que la production de littérature dans les domaines scientifiques connaît une telle croissance qu’il est devenu humainement difficile d’analyser les articles pertinents noyés dans la masse de documents. La reconnaissance d'entité nommées permet donc d'aider à sélectionner les documents contenant des noms appartenant à une certaine entité (comme la catégorie des protéines par exemple). Les systèmes d'extractions de connaissance, qui servent à réunir et mettre à disposition de manière structurée l'information disponible dans la littérature, sont également alimentés par la reconnaissance d'entités nommées.