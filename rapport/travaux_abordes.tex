Nous présentons dans ce rapport ainsi 3 méthodes de reconnaissance d'entité nommée qui sont appliquées sur le corpus GENIA. Dans le chapitre 2, nous présentons une vue générale du système. Chaque élément du système est ensuite détaillé (pré-traitements, extraction des descripteurs) ainsi que les 3 méthodes de classification (NB, HMM et CRF). Les expériences sont présentées dans le chapitre 3. Dans ce chapitre, nous rappelons brièvement le contenu et la structure du corpus GENIA. Les tests ne sont appliqués que sur une partie des entités nommées du corpus. Nous expliquons donc quelles entités nommées ont été sélectionnées pour l'évaluation du système. Enfin, nous présentons les différentes configurations des systèmes testés ainsi que leur résultats. Finalement, le dernier chapitre expose les comparaisons et critiques des configurations testées.