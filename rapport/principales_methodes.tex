\noindent\gr{Méthodes Rules-Based}\\

Historiquement, les premières méthodes étaient basées sur des règles ("Rules-Based") élaborées soit de manière automatique, soit écrites "à la main". Le principe se base sur des paires $(pattern, action)$ où un $pattern$ correspond généralement à un une expression régulière. Lorsqu'un token (ou une séquence de tokens) correspond à un $pattern$, une $action$ associée est exécutée. Cette $action$ correspond à l'étiquetage des tokens (entité, début ou fin de l'entité par exemple). Bien que très efficaces, ces approches nécessitent, dans le cas des règles écrites à la main, un expert à la fois de la langue et du domaine, rendant la technique très spécifique et très coûteuse en temps. Pour le cas des règles obtenues de manière automatique, elles souffrent d'un manque de précision \cite{Mladenic2017}. On trouve également d'autres inconvénients impactant par exemple la robustesse du système de reconnaissance. En effet, lorsque de nouvelles données nécessitent de nouvelles règles, il faut alors mettre à jour la table de règles.\\

\noindent\gr{Méthodes Statistiques}\\

Les approches statistiques permettent de pallier certains inconvénients des méthodes Rules-Based car les "règles" sont apprises sur les données et non à la main. Pour la reconnaissance d'entité nommées, on distingue dans ces méthodes d'apprentissage statistique principalement 2 types d'approches: semi-supervisées et supervisées. Les deux types d'approches font usage des données étiquetées. Une partie de ces données sert à entraîner le modèle de classification, puis les données restantes servent à évaluer les performances du modèle.\\
Dans les méthodes semi-supervisées, le jeu d'entraînement est constitué de données étiquetées et non-étiquetées. Nous ne traiterons pas le cas des méthodes semi-supervisées dans ce rapport. On peut néanmoins citer les approches par Boostrapping \cite{Thenmalar} et par Co-training \cite{Cotraining}.\\
Pour le cas des méthodes supervisées, les données d'entraînement contiennent uniquement des données étiquetées. De nombreuses approches en apprentissage supervisé ont été développées basées sur les arbres de décision, modèles de Maximum d'Entropie, SVM, Réseaux de neurones, Bayésien Naïf, HMM, CRF ... (la liste n'est pas exhaustive).\\
Dans ce rapport, nous ne présenterons que les méthodes Bayésien Naïf (NB), Modèles de Markov Cachés (HMM) et Conditional Random Field (CRF).