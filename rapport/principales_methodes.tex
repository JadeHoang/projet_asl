\noindent\gr{Méthodes Rules-based}\\

Historiquement, les premières méthodes étaient basées sur des règles élaborées soit de manière automatique, soit écrites "à la main". Le principe se base sur des paires $pattern, action$ où un $pattern$ correspond généralement à un une expression régulière. Lorsqu'un token (ou une séquence de tokens) correspond à un $pattern$, une $action$ est exécutée. Cette $action$ correspond à l'étiquetage des tokens (entité, début ou fin de l'entité par exemple). Bien que très efficaces, ces approches nécessitent, dans le cas des règles écrites à la main, un expert à la fois de la langue et du domaine, rendant la technique très spécifique et très coûteuse en temps. Pour le cas des règles obtenues de manière automatique, elles souffrent d'un manque de précision \cite{Mladenic2017}.\\

\noindent\gr{Méthodes Statistiques}\\

L
Machine-learning techniques
Supervised-Learning : (non exhaustif) SVM, CRF, HMM, Naïve Bayesian, Neural Network, Decision Tree, Maximum Entropy Model
Semi-Supervised-Learning : (données labélisées et non labélisées) Boot-strapping, Co-Training, 
Unsupervised-Learning : méthodes de clustering.
Dans notre étude, nous ne nous concentrons que sur 3 méthodes d’apprentissage supervisé.

Développement de méthodes statistiques qui ont très rapidement dépassé les résultats des méthodes basées sur les règles.}