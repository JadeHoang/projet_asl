\red{
Corpus GENIA : corpus développé pour le projet GENIA dont l’objectif est d’explorer les méthodes d’extraction d’information et de text mining spécifiques au domaine de la science médicale. Le corpus peut ainsi servir de référence pour la communauté scientifique dans les tâches citées ci-dessus. Le corpus GENIA est composé d’un sous-ensemble d’éléments d’articles de la base de données Medline spécifiques aux réactions biologiques impliquées dans les « transcriptions factors in human bloods cells ». Ainsi pour chaque article, seuls les titres et résumés ont été collectés à partir de requêtes sur l’interface web de PubMed.

Pré-traiter les documents du corpus xml de façon à enlever les méta-données et à représenter
les textes sous forme numérique selon les besoins des classifieurs

}