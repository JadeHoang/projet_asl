\red{Problème : en se basant sur des données annotées, on essaie de retrouver des entités nommées dans de nouveaux documents.

Solution stupide : Faire une recherche des formes lémmatisées à partir d'un dictionnaire obtenue sur le jeu d'entrainement.\\

==>Impossible de retrouver de nouveaux mots\\
==>On risque de classer un mot figurant dans les mots annotés mais ayant un sens différent car dans un contexte différent.\\

Plusieurs méthodes ont été proposées:


Rules-based techniques
Premières méthodes basées sur des règles faites à la main. Ces approches sont très efficaces. Cependant le problème est qu’elles nécessitent un expert à la fois de la langue et du domaine rendant la technique très spécifique et très coûteuse en temps.
Règles comme « si ‘X’ est précédé d’une préposition ‘Y’

Machine-learning techniques
Supervised-Learning : (non exhaustif) SVM, CRF, HMM, Naïve Bayesian, Neural Network, Decision Tree, Maximum Entropy Model
Semi-Supervised-Learning : (données labélisées et non labélisées) Boot-strapping, Co-Training, 
Unsupervised-Learning : méthodes de clustering.
Dans notre étude, nous ne nous concentrons que sur 3 méthodes d’apprentissage supervisé.

Développement de méthodes statistiques qui ont très rapidement dépassé les résultats des méthodes basées sur les règles.}