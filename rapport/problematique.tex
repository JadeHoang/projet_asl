\red{Le problème considéré est une tâche de catégorisation pour laquelle il faut attribuer une classe à un mot. Nous disposons pour cela d'un corpus de données annotées dont une partie doit servir à alimenter le système de reconnaissance tandis que l'autre sert à l'évaluer.\\
Pour résoudre ce problème, on pourrait penser que la solution triviale consiste à comparer le mot à classifier à des dictionnaires de mots obtenus dans le jeu de données labellisées. Chaque dictionnaire correspondant à une classe, si le mot existe dans un dictionnaire alors il serait attribué à la classe correspondante. Une telle solution trouve malheureusement rapidement de nombreuses limitations. Comment classifier un mot qui n'existe dans aucun des dictionnaires (c'est à dire jamais observé dans le jeu d’entraînement) ? Par ailleurs, un même mot peut recouvrir plusieurs sens et donc appartenir à une classe différente selon le contexte. Il s'agit ici d'un problème d'ambiguïté. Le problème n'étant pas si simple, plusieurs méthodes ont été proposées.